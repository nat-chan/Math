\documentclass{ltjsarticle}
\usepackage[top=3truemm,bottom=3truemm,left=3truemm,right=3truemm]{geometry}
\usepackage{graphicx}
%\usepackage{subfiles}
%\usepackage{lastpage}
%\usepackage{fancyhdr}
\usepackage{amsmath}
\usepackage{amssymb}
\usepackage{ascmac}
\usepackage{url}       % \urlでエスケープなしでurlを表示
%\usepackage{listings}  % ソースコード
%\usepackage{color}
%\usepackage{here}      % figureで[H]を使えるようにする
\usepackage{mathtools} % 参照している式番号のみを表示
\mathtoolsset{showonlyrefs=true}
%\usepackage{bm} % \bm{}でbold,italicでベクトルを表す
%\lstset{
%	numbers=left,
%	numbersep=1em,
%%	numberstyle=\tiny\color{red}\noaccsupp,
%	frame=single,
%%	framesep=\fboxsep,
%%	framerule=\fboxrule,
%%	rulecolor=\color{red},
%%	xleftmargin=\dimexpr\fboxsep+\fboxrule\relax,
%%	xrightmargin=\dimexpr\fboxsep+\fboxrule\relax,
%	breaklines=true,
%	columns=flexible,
%	basicstyle=\ttfamily\footnotesize,
%	tabsize=4
%}
%\graphicspath{{../images/}} %実は相対パスで指定してる
%\nonumber
%\renewcommand{\headrulewidth}{0pt} %\headrulewidthは未定義
%\renewcommand{\footrulewidth}{0pt}
\usepackage{luatexja-fontspec}

\setmainfont[Ligatures=TeX]{TeXGyreTermes}
\setsansfont[Ligatures=TeX]{TeXGyreHeros}

\setmainjfont[BoldFont=IPAexGothic]{IPAexMincho}
\setsansjfont{IPAexGothic}

\renewcommand{\thesubsection}{(\thesection--\arabic{subsection})}
\renewcommand{\thesubsubsection}{\thesubsection}
\newcommand{\↊}{\rotatebox[origin=c]{180}{2}}
\newcommand{\↋}{\rotatebox[origin=c]{180}{3}}
\begin{document}
\begin{gather}
	\|x,y\|=12\left|\log_2\frac{x}{y} \right| \\
	12進数:0,1,2,3,4,5,6,7,8,9,\chi χ=\↊,\varepsilon ε=\↋ \\
	メジャースケール:0,2,4,5,7,9,11,12\\
	\begin{array}{cccccccc}
		C&D&E&F&G&A&B&C\\
		0&2&4&5&7&9&11&12
	\end{array}\\
	\begin{array}{ccccccccccccc}
		C   & C^\#  & D     & D^\#  & E     & F       & F^\#  & G       & G^\#  & A     & A^\#  & B     & C\\
		0   & 1     & 2     & 3     & 4     & 5       & 6     & 7       & 8     & 9     & 10    & 11    & 12\\
		1度 & 短2度 & 長2度 & 短3度 & 長3度 & 完全4度 & 増4度 & 完全5度 & 短6度 & 長6度 & 短7度 & 長7度 & 8度\\
		1:1 &       &       &       & 4:5   & 3:4     &       & 2:3     &       &       &       &       & 1:2
	\end{array}\\
	構成音のナンバーを昇順に\{a_i\}_{i=1}^{n}\in \mathbb{N}^{\ast}\\
	(a_1 |\{a_{i+1}-a_{i+1}\}_{i=1}^{n-1})\\
	C_△→(0|43)\\
	C_m→(0|34)\\
	C_{aug}→(0|44)\\
	C_{△7}→(0|434)\\
	C_{m7}→(0|343)\\
	平行移動:\\
	C^\#_△→(1|43)\\
	A_m→(9|34)\\
	転回形:\\
		(b_1 | b_2 b_3\ldots b_n)→ \left(b_1+b_2 \mid b_3\ldots b_n, b_1+12-\sum_{i=2}^n b_i\right)
\end{gather}
ノートナンバー:真ん中のCが60

440hz:中央A
\end{document}
