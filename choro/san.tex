\documentclass[./choro]{subfiles}
%\setcounter{section}{1}
\renewcommand{\baselinestretch}{0}
\begin{document}
%{{{ フォントサイズ
%\fontsize{20pt}{0pt}\selectfont
%\large
%\tiny
%\scriptsize
\footnotesize
%\small
%\normalsize
%\large
%\Large
%\LARGE
%\huge
%\Huge
%}}}
%$A\rm{A}\bf{A}_1$
%$\mathbb{x}$

%{{{multicols
%\begin{multicols}{3}
%    \subfile{toiori}
%    \begin{align}
%        \gA{1} &= \gaU{1} \A{1} \gU{1} &
%        \gA{2} &= \gaU{2} \A{2} \gU{2} \\
%               &= diag(\lambda(\A{1})) &
%               &= diag(\lambda(\A{2})) \\
%        \gK{1} &= \gaU{1} \K{1}        &
%        \gK{2} &= \gaU{2} \K{2}        \\
%    \end{align}
%    \subfile{kityaku}
%\end{multicols}
%}}}

\begin{minipage}{0.3\hsize}
    \subfile{toiori}
\end{minipage}
\begin{minipage}{0.35\hsize}
    \vspace{2cm}
    \begin{itembox}[l]{対角化}
        $A_1,A_2$はエルミート行列なのでユニタリ行列をとってきて対角化できる
        \begin{align*}
            \gA{1} &= \aU{1} \A{1} \U{1} &
            \gA{2} &= \aU{2} \A{2} \U{2} \\
                   &= diag(\lambda(\A{1})) &
                   &= diag(\lambda(\A{2})) \\
        \end{align*}
        あたらしく$K$を取り直す
        \begin{align*}
            \gK{1} &= \aU{1} \K{1}        &
            \gK{2} &= \aU{2} \K{2}        \\
        \end{align*}
    \end{itembox}
        ---------------------------------------------------------------------▶
    \begin{itembox}[l]{問題の同値性}
        \begin{align*}
            &               & \aK{1} \K{1} &+
                              \aK{2} \K{2} = I \\
            &\Leftrightarrow& \aK{1} \U{1}\aU{1} \K{1} &+
                              \aK{2} \U{2}\aU{2}\K{2} = I \\
            &\Leftrightarrow& (\aU{1}\K{1})^{\ast} (\aU{1} \K{1}) &+ 
                              (\aU{2}\K{2})^{\ast} (\aU{2} \K{2}) = I \\
            &\Leftrightarrow& \gaK{1} \gK{1} &+ \gaK{2} \gK{2} = I
        \end{align*}
        \begin{align*}
            &  & \aK{1} \A{1} \K{1} &+ \aK{2} \A{2} \K{2}  \\
            &= & \aK{1} \U{1}\aU{1} \A{1} \aU{1}\U{1} \K{1} &+
                 \aK{2} \U{2}\aU{2} \A{2} \aU{2}\U{2} \K{2}  \\
            &= & \aK{1} \U{1} \gA{1}\U{1} \K{1} &+
                 \aK{2} \U{2} \gA{2}\U{2} \K{2}  \\
            &= & (\aU{1}\K{1})^{\ast} \gA{1} (\aU{1} \K{1}) &+ 
                 (\aU{2}\K{2})^{\ast} \gA{2} (\aU{2} \K{2}) \\
            &= & \gaK{1} \gA{1} \gK{1} &+ \gaK{2} \gA{2} \gK{2} 
        \end{align*}
    \end{itembox}
\end{minipage}
\begin{minipage}{0.3\hsize}
    \subfile{kityaku}
\end{minipage}
\end{document}
